\documentclass[11pt]{article}
\usepackage{rldm25}
\usepackage{palatino}
\usepackage{graphicx}
\usepackage{amsmath}
\usepackage[T1]{fontenc}
\usepackage{lmodern}  % Fuente moderna con soporte completo
\renewcommand{\bfdefault}{b} % Asegura que las negritas se activen


\title{Formatting Instructions for RLDM}

\author{
  Dorsal S. Raphe % Nombre del autor
      \\       Department of Computer Science % Afiliación
      \\         Cranberry-Lemon University, Pittsburgh, PA
15213 % Dirección
                  \\ \texttt{raphe@cs.cranberry-lemon.edu} % Email
        \And
    Coauthor % Nombre del autor
      \\       Affiliation % Afiliación
      \\         Address % Dirección
                  \\ \texttt{email@example.com} % Email
        \And
  }


\begin{document}

\maketitle

\begin{abstract}
The \emph{title} should be a maximum of 100 characters.

The \emph{abstract} should be a maximum of 2000 characters of text,
including spaces (no figure is allowed). You will be asked to copy this
into a text-only box; and it will appear as such in the conference
booklet. Use 11-point type, with a vertical spacing of 12 points. The
word \textbf{Abstract} must be centered, bold, and in point size 12. Two
line spaces precede the abstract.
\end{abstract}

\keywords{RLDM, example, formatting}

\newpage

\section{Submission of papers to
RLDM}\label{submission-of-papers-to-rldm}

RLDM requires electronic submissions. This year's electronic submission
site is:

\begin{center}
   https://cmt3.research.microsoft.com/RLDM2017/
\end{center}

Please read the instructions below, and follow them faithfully. Note
that there is also a template \texttt{rldm.rtf} for Microsoft Word,
which is available from the website below.

\subsection{Style}\label{style}

Papers to be submitted to RLDM must be prepared according to the
instructions presented here. Papers consist of a \emph{title}, which has
a maximum of 100 characters, an \emph{abstract}, which is a maximum of
2000 characters, up to five key words, and an \emph{extended abstract},
which starts on the second page, and can be between one and four pages.
Figures and references should be included in the latter.

Authors preferring \LaTeX{} are requested to use the RLDM \LaTeX{} style
files obtainable at the RLDM website at:

\begin{center}
   http://www.rldm.org/
\end{center}

The file \texttt{rldm.pdf} contains these instructions and illustrates
the various formatting requirements your RLDM paper must satisfy. There
is a \LaTeX{} style file called \texttt{rldmsubmit.sty}, and a \LaTeX{}
file \texttt{rldm.tex}, which may be used as a ``shell'' for writing
your paper. All you have to do is replace the author, title, abstract,
keywords, acknowledgements and text of the paper with your own. The file
\texttt{rldm.rtf} is provided as an equivalent shell for Microsoft Word
users.

\section{General formatting
instructions}\label{general-formatting-instructions}

The paper size for RLDM is ``US Letter'' (rather than ``A4''). Margins
are 1.5cm around all sides. Use 11-point type with a vertical spacing of
12 points. Palatino is the preferred typeface throughout. Paragraphs are
separated by 1/2 line space, with no indentation.

Paper title is 17-point, initial caps/lower case, bold, centered between
2 horizontal rules. Top rule is 4 points thick and bottom rule is 1
point thick. Allow 0.6cm space above and below title to rules.

The lead author's name is to be listed first (left-most), and the
co-authors' names (if different address) are set to follow. If there is
only one co-author, list both author and co-author side by side.

\section{Preparing PostScript or PDF
files}\label{preparing-postscript-or-pdf-files}

Please prepare PostScript or PDF files with paper size ``US Letter''.
The -t letter option on dvips will produce US Letter files.

\end{document}
